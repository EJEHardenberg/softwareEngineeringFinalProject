\documentclass[12pt]{IEEEtran}

\usepackage[english]{babel}
\usepackage[utf8x]{inputenc}
\usepackage{amsmath}
\usepackage{graphicx}

\title{Requirements and Specifications}
\author{Bryan Ceberio-Lucas \and Ethan Eldridge \and Peter LeBlanc \and Phelan Vendeville }

\begin{document}
\maketitle

\begin{abstract}
	This document contains the specifications of CS 205 Software  Engineering's final project, an implementation of Rat-a-tat Cat. These standards and requirements will be followed by all team members. The 			following terms and descriptions must be clear to all members so that the system is a cohesive and comprehendable system.
\end{abstract}

\tableofcontents

\section{Terms and Definitions}
\label{sec:TermsDefinitions}
	\begin{description}
		\item[Must] If a specification uses the word Must, it is mandatory that all team members follow this requirement. E.g.  \textit{The System \textbf{must} handle all possible URLs and direct the user's to an 				appropriate page.} 
		\item[Shall] If a specification uses the word Shall, then the System must respond to the specification in the detailed way. E.g. \textit{The system \textbf{shall} perform operations in a timely manner and 				no operation will take more than 10 seconds}
		\item[Gantt] A bar graph used to visualize a project schedule
		\item[Glow] To glow is to surround an object with a faint highlight that indicates that the User may interact with this object.
		\item[State] The System's internal state is kept using a Stack of strings that indicate the current and next state of the System, this collection is refered to as the State and can be pushed, popped, and 				peeked.
		\item[Magic Numbers] \hspace{4em} Hard-coded numerical constants 
	\end{description}

\section{Introduction}

	Phelans awesome introduction goes here

\section{Introduction}
\label{sec:introduction}

\section{Scope and Purpose}
\label{sec:scope}

\subsection{Scope}
\label{subsec:scope}

What is the scope of our project

\subsection{Purpose}
\label{subsec:purpose}

This is where the purpose of our project goes

\section{Functional Requirements}
\label{sec:funcReq}
	The functional requirements of this project are specified by the Coding Standards in \S \ref{subsec:coding}, Version Control standards in \S \ref{subsec:git}, directory and game Architecture in \S 				\ref{subsec:arch}, Artificial Intelligence logical overview in \S \ref{subsec:ai}, and Database Design images and naming conventions in \S \ref{subsec:dbdesign}.


\subsection{Coding Standards}
\label{subsec:coding}

	The following standards must be followed by all team members. By defining these standards all code will be readable for all members, and no discrepencies between conventions will occur. Each team member is 		responsible for keeping to these standards, and submission of code not keeping to these standards will come under review and the format shall be adjusted accordingly. 

	\bfseries Naming conventions \mdseries

	\begin{itemize}
		\item Variable names must be camelcase, descriptive, and self documenting
		\item Class names must begin with a capital letter and use camelcase 
		\item Database table names must begin with a capital letter and use camelcase
		\item Database table names should be short, one word where ever possible
		\item Directories must be lowercase and without spaces
		\item File names must be lowercase and without spaces
		\item File names for card images must be the value of the card, or 10-12 for power cards.
		\item All images should end in .png and be of that format
		\item CSS class names must be self-documenting
		\item CSS class names must be camel case
		\item Constants in any form must be all uppercase with underscores between natural breaks
		\item Git tagging must follow the convention of version\_x.y, x must be the major release number, y the minor release number
		\item The team leaders repository should be refered to as mainline during remote declaration
	\end{itemize}

	\bfseries Commenting Conventions \mdseries

	At the beginning of each function or class there must be a comment section within triple qoutes defining the following:
	\begin{itemize}
		\item Description of function or class
		\item A list of parameters and types of each
		\item A brief description of the return type of the function
	\end{itemize}

	At the top of each code file there must be a comment section with the following information:
	\begin{itemize}
		\item A description of the file's purpose and intent
		\item A list of the functions or classes defined within the file
		\item The date the file was made
		\item The date of the most recent revision
		\item A list of authors or modifiers of the file. 
	\end{itemize}

	Within HTML each ending $<div>$ should have a comment indicating the id of the opening tag. CSS comments should be used to partition style sheet files into managable and well ordered blocks of style. It 			must be easy to determine which content is affected by the style by simpling reading through the comments.

	\bfseries General Conventions and Guidelines  \mdseries

	\begin{itemize}
		\item Conditional statements that involve more than a single variable must use parenthesis
		\item All sensitive information should be passed through posting whenever possible
		\item HTML/CSS should pass validation tests and be well formed and self documenting
		\item Global Variables should only be used when neccesary
		\item Magic Numbers should be avoided whenever possible 
		\item Formal specifications should be made available using the shared Google Drive or through the mainline Git repository
	\end{itemize}

\subsection{Version Control}
\label{subsec:git}

	The version control used to maintain the source code for this System is Git. The following standards must be followed by all team members in order to maintain proper source code management.
	\begin{itemize}
		\item Git commits must be descriptive and verbose
		\item When merging feature and component branches to dev or mainline the option --no-ff must be used
		\item The Git tagging system must be used to maintain stable release checkpoints
		\item The master branch of the mainline repository must be functional
		\item Rebasing commit history is forbidden if the history has been pushed to a remote repository
		\item A team member resolving merge conflicts must ensure the merge is agreeable to both their and the incoming code
		\item All members must have their global config setup with email and name for source code tracking purposes
	\end{itemize}
	

\subsection{Architecture and Structure}
\label{subsec:arch}

	The System shall use the Google App Engine (GAE) and Jinja templating systems to function. The model-view-controller paradigm will be implemented, in this instance the model will be the database backend from 		GAE. The view will be comprised of Jinja templates, Javascript/JQuery, and CSS. The python files used by the GAE will handle all control information and dictate the flow control of the System. The project must                                 	be organized, and the directory structure will be as follows:

	\begin{itemize}
		\item \texttt{python/}
		\item \texttt{config/}
		\item \texttt{templates/}
		\item \texttt{css/}
		\item \texttt{scripts/}
		\item \texttt{images/}
		\item \texttt{userimages/}
		\item \texttt{sounds/}
	\end{itemize}

	These directory names are self explanatory besides the difference between \texttt{images} and \texttt{userimages}. The \texttt{userimages} directory is a location where user uploaded images may be stored, 		this directory is kept seperate for security purposes. 

	Each URL handled by the GAE framework and our configuration files is mapped to a python controller in a many to one relationship. The URL to python controller mapping is defined as follows:
	
	\begin{center}
		\begin{tabular}{| c | c |}\hline
			\multicolumn{1}{|c}{URL} & \multicolumn{1}{c|}{ Handler Name }\\\hline
			 \texttt{/}			& MasterControlProgram\\\hline
			 \texttt{/scores} 		& ScoresHandler\\\hline
			 \texttt{/game} 		& GameHandler\\\hline
			 \texttt{/playerinfo}	& PlayerInfoHandler\\\hline
			 \texttt{/characterchoice} & CharacterHandler\\\hline
			 \texttt{/difficulty}		& DifficultyHandler\\\hline
		\end{tabular}
	\end{center}

\subsection{Artificial Intelligence}
\label{subsec:ai}



\subsection{Database Design}
\label{subsec:dbdesign}

\section{Non-Functional Requirements}
\label{sec:nonFuncReq}

\subsection{User Interface}
\label{subsec:ui}

pretty pictures and descriptions galore

\subsection{Game Play}
\label{subsec:gameplay}

	This is where the storyboarding stuff goes

\subsection{Character Design and Concept Art}
\label{subsec:cdesign}

\subsection{Timeline and Delivery}
\label{subsec:timeline}

	This is where timeline and due dates go as well as what has to go into each part

\section{Test Cases}
\label{sec:test}

\section{Summary}
\label{sec:summary}

	Overall summary


\end{document}