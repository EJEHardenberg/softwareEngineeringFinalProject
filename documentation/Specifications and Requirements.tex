\documentclass[12pt]{IEEEtran}

\usepackage[english]{babel}
\usepackage[utf8x]{inputenc}
\usepackage{amsmath}
\usepackage{graphicx}

\title{Requirements and Specifications}
\author{Bryan Ceberio-Lucas \and Ethan Eldridge \and Peter LeBlanc \and Phelan Vendeville }

\begin{document}
\maketitle

\begin{abstract}
	This document contains the specifications of CS 205 Software  Engineering's final project, an implementation of Rat-a-tat Cat. These standards and requirements will be followed by all team members. The 			following terms and descriptions must be clear to all members so that the system is a cohesive and comprehendable system.
\end{abstract}

\tableofcontents

\section{Terms and Definitions}
\label{sec:TermsDefinitions}
	\begin{description}
		\item[Must] If a specification uses the word Must, it is mandatory that all team members follow this requirement. E.g.  \textit{The System \textbf{must} handle all possible URLs and direct the user's to an 				appropriate page.} 
		\item[Shall] If a specification uses the word Shall, then the System must respond to the specification in the detailed way. E.g. \textit{The system \textbf{shall} perform operations in a timely manner and 				no operation will take more than 10 seconds}
		\item[Gantt] A bar graph used to visualize a project schedule
		\item[Glow] To glow is to surround an object with a faint highlight that indicates that the User may interact with this object.
		\item[State] The System's internal state is kept using a Stack of strings that indicate the current and next state of the System, this collection is refered to as the State and can be pushed, popped, and 				peeked.
		\item[Magic Numbers] \hspace{4em} Hard-coded numerical constants 
		\item[Knock] The button press that determines a round is over and the cards should be overturned when gameplay returns to the user who knocked.
	\end{description}

\section{Introduction}
\label{sec:introduction}

	Rat-a-Tat-Cat is an award winning children’s card game produced by Gamewright. Using a set of easy to learn and intuitive rules, Rat-a-Tat-Cat teaches memory, simple math, and probability skills to players over the course of the game. 
	
	It is our goal to create a software implementation of this game based on the desires of our client Jason Hibbeler, and falling within the basic constraints of the standard game rules. This implementation will be a web-based, interactive and graphical representation of Rat-a-Tat-Cat, providing the user with an experience equal to or exceeding that of the physical game.

\section{Scope and Purpose}
\label{sec:scope}

\subsection{Scope}
\label{subsec:scope}

	High level scope for this game includes a set of several features to be delivered over the course of several weeks in the University of Vermont Spring Semester. This software product and its features is restricted to academic use and is not to be shipped for profit. The receipt of these features will indicate a fulfillment of obligation on the part of the software engineering team to the client. These high level deliverables include this technical specifications document, comprehensive product testing data documents, and a completed, playable game of Rat-a-Tat-Cat based on the constraints provided by the client.

\subsection{Purpose}
\label{subsec:purpose}

	This document will be used to delineate the behavior, structure and requirements of the Rat-a-Tat-Cat game system, including both functional and nonfunctional elements. This document has been written to be read, and as such will serve as a guide to both developers and the client. It shall provide a followable rubric to the software engineering team for enumeration of executable and deliverable expectations, and give the client a set of expectations to be fulfilled. 
	
	We shall begin the specifications document with an exploration of the project’s functional requirements, then continue with its nonfunctional stipulations. We will outline a series of test cases to ascertain the completion of such exigencies, and conclude with a brief summary of the document.

\section{Functional Requirements}
\label{sec:funcReq}
	The functional requirements of this project are specified by the Coding Standards in \S \ref{subsec:coding}, Version Control standards in \S \ref{subsec:git}, directory and game Architecture in \S 				\ref{subsec:arch}, Artificial Intelligence logical overview in \S \ref{subsec:ai}, and Database Design images and naming conventions in \S \ref{subsec:dbdesign}.


\subsection{Coding Standards}
\label{subsec:coding}

	The following standards must be followed by all team members. By defining these standards all code will be readable for all members, and no discrepencies between conventions will occur. Each team member is 		responsible for keeping to these standards, and submission of code not keeping to these standards will come under review and the format shall be adjusted accordingly. 

	\bfseries Naming conventions \mdseries

	\begin{itemize}
		\item Variable names must be camelcase, descriptive, and self documenting
		\item Class names must begin with a capital letter and use camelcase 
		\item Database table names must begin with a capital letter and use camelcase
		\item Database table names should be short, one word where ever possible
		\item Directories must be lowercase and without spaces
		\item File names must be lowercase and without spaces
		\item File names for card images must be the value of the card, or 10-12 for power cards.
		\item All images should end in .png and be of that format
		\item CSS class names must be self-documenting
		\item CSS class names must be camel case
		\item Constants in any form must be all uppercase with underscores between natural breaks
		\item Git tagging must follow the convention of version\_x.y, x must be the major release number, y the minor release number
		\item The team leaders repository should be refered to as mainline during remote declaration
	\end{itemize}

	\bfseries Commenting Conventions \mdseries

	At the beginning of each function or class there must be a comment section within triple qoutes defining the following:
	\begin{itemize}
		\item Description of function or class
		\item A list of parameters and types of each
		\item A brief description of the return type of the function
	\end{itemize}

	At the top of each code file there must be a comment section with the following information:
	\begin{itemize}
		\item A description of the file's purpose and intent
		\item A list of the functions or classes defined within the file
		\item The date the file was made
		\item The date of the most recent revision
		\item A list of authors or modifiers of the file. 
	\end{itemize}

	Within HTML each ending $<div>$ should have a comment indicating the id of the opening tag. CSS comments should be used to partition style sheet files into managable and well ordered blocks of style. It 			must be easy to determine which content is affected by the style by simpling reading through the comments.

	\bfseries General Conventions and Guidelines  \mdseries

	\begin{itemize}
		\item Conditional statements that involve more than a single variable must use parenthesis
		\item All sensitive information should be passed through posting whenever possible
		\item HTML/CSS should pass validation tests and be well formed and self documenting
		\item Global Variables should only be used when neccesary
		\item Magic Numbers should be avoided whenever possible 
		\item Formal specifications should be made available using the shared Google Drive or through the mainline Git repository
	\end{itemize}

\subsection{Version Control}
\label{subsec:git}

	The version control used to maintain the source code for this System is Git. The following standards must be followed by all team members in order to maintain proper source code management.
	\begin{itemize}
		\item Git commits must be descriptive and verbose
		\item When merging feature and component branches to dev or mainline the option --no-ff must be used
		\item The Git tagging system must be used to maintain stable release checkpoints
		\item The master branch of the mainline repository must be functional
		\item Rebasing commit history is forbidden if the history has been pushed to a remote repository
		\item A team member resolving merge conflicts must ensure the merge is agreeable to both their and the incoming code
		\item All members must have their global config setup with email and name for source code tracking purposes
	\end{itemize}
	

\subsection{Architecture and Structure}
\label{subsec:arch}

	The System shall use the Google App Engine (GAE) and Jinja templating systems to function. The model-view-controller paradigm will be implemented, in this instance the model will be the database backend from 		GAE. The view will be comprised of Jinja templates, Javascript/JQuery, and CSS. The python files used by the GAE will handle all control information and dictate the flow control of the System. The project must                                 	be organized, and the directory structure will be as follows:

	\begin{itemize}
		\item \texttt{python/}
		\item \texttt{config/}
		\item \texttt{templates/}
		\item \texttt{css/}
		\item \texttt{scripts/}
		\item \texttt{images/}
		\item \texttt{userimages/}
		\item \texttt{sounds/}
	\end{itemize}

	These directory names are self explanatory besides the difference between \texttt{images} and \texttt{userimages}. The \texttt{userimages} directory is a location where user uploaded images may be stored, 		this directory is kept seperate for security purposes. 

	Each URL handled by the GAE framework and our configuration files is mapped to a python controller in a many to one relationship. The URL to python controller mapping is defined as follows:
	
	\begin{center}
		\begin{tabular}{| c | c |}\hline
			\multicolumn{1}{|c}{URL} & \multicolumn{1}{c|}{ Handler Name }\\\hline
			 \texttt{/}			& MasterControlProgram\\\hline
			 \texttt{/scores} 		& ScoresHandler\\\hline
			 \texttt{/game} 		& GameHandler\\\hline
			 \texttt{/playerinfo}	& PlayerInfoHandler\\\hline
			 \texttt{/characterchoice} & CharacterHandler\\\hline
			 \texttt{/difficulty}		& DifficultyHandler\\\hline
		\end{tabular}
	\end{center}

\subsection{Artificial Intelligence}
\label{subsec:ai}

	The artificial intelligence (AI) within the System is stochastic in nature. A difficulty modifier passed via the games start up parameters influences the reliability of the AI's memory via a decay rate. Each level of 		difficutly effects the decay rate of the memory as well as the baseline of remembrance. The AI is not all knowing, and keeps estimates of both its and the opponents cards. Using these estimates a strategy can 		be used to determine when to knock and when to use power cards. For each of the four cards in the AI's hand a rate of memory decay is kept. This value determines how well the AI remembers the value of the 		card and is decremented each round according to the decay rate value stored within the AI.

	\begin{description}
		\item[Swap Strategy] \hspace{3em} Using an internal estimate, the AI selects it's highest known card and switches it with the players lowest known card. The AI can determine what the players cards are 			based	on if the Player has drawn from the discard pile or swapped with one of the AI's known cards. In the case of no cards being known -- whether in selection of its own highest card or the users -- a 			random card is chose. 
		\item[Peek Strategy] \hspace{3em} The AI will always select a card it does not know to peek. If all the values are known, the AI will select the card with the lowest decayed value, and that value will be 			reset to 1.
		\item[Knocking] \hspace{3em}  The AI will knock when it has reasonable confidence that its internal estimate of its own score is higher than the players.
	\end{description}

	The AI will also keep track of internal statistics to send to the database later. This is primarily a logging function and is an optional part of the AI implementation.

\subsection{Database Design}
\label{subsec:dbdesign}

\section{Non-Functional Requirements}
\label{sec:nonFuncReq}

\subsection{User Interface}
\label{subsec:ui}

pretty pictures and descriptions galore

\subsection{Game Play}
\label{subsec:gameplay}

	This is where the storyboarding stuff goes

\subsection{Character Design and Concept Art}
\label{subsec:cdesign}

\subsection{Timeline and Delivery}
\label{subsec:timeline}

	This is where timeline and due dates go as well as what has to go into each part

\section{Test Cases}
\label{sec:test}

\section{Summary}
\label{sec:summary}

	It is our desire that this implementation of the Rat-a-Tat-Cat card game capture the spirit and mechanics of the original game while adapting it to an interactive, web-driven and graphically pleasing format. This software implementation of Rat-a-Tat-Cat should be stable, robust, and eminently usable, providing players with an intuitive and consistently enjoyable gameplay experience in a natural interface. This specifications document is intended to provide the necessary framework to both implement and measure the success of these requirements, as well as set followable engineering standards to facilitate the creative process and satisfy the client’s inquiries and expectations.


\end{document}